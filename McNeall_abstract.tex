
\documentclass[esd, manuscript]{copernicus} % uncomment to see what the 2 column final paper will look like.

\begin{document}

\title{The impact of structural error on parameter constraint in a climate model}

% \Author[affil]{given_name}{surname}

\Author[1]{Doug}{McNeall}
\Author[2]{Jonny}{Williams}
\Author[1]{Ben}{Booth}
\Author[1]{Richard}{Betts}
\Author[3]{Peter}{Challenor}
\Author[1]{Andy}{Wiltshire}
\Author[1]{David}{Sexton}

\affil[1]{Met Office Hadley Centre, FitzRoy Road, Exeter, EX1 3PB UK}
\affil[2]{NIWA, 301 Evans Bay Parade, Hataitai, Wellington 6021, New Zealand}
\affil[3]{University of Exeter, North Park Road, Exeter EX4 4QE UK}


\maketitle

We use observations of forest fraction to constrain carbon cycle and land surface input parameters of the reduced resolution global climate model, FAMOUS. Using a history matching approach along with a computationally cheap statistical proxy (emulator) of the climate model, we compare an ensemble of simulations of forest fraction with observations, and rule out parameter settings where the forests are poorly simulated. 

Regions of parameter space where FAMOUS best simulates the Amazon forest fraction are incompatible with the regions where FAMOUS best simulates other forests, indicating a structural error in the model. Using observations of the Amazon forest to constrain input parameters leads to very different conclusions about the acceptable values of input parameters than using the other forests.

We characterise the structural model discrepancy, and explore the consequences of ignoring it in a history matching exercise. We use sensitivity analysis to find the parameters which have most impact on simulator error.  We use the emulator to simulate the forest fraction at the best set of parameters implied by matching the model to the Amazon, and to other major forests in turn. We can find parameters that lead to a realistic forest fraction in the Amazon, but using the Amazon alone to tune the simulator would result in a significant overestimate of forest fraction in the other forests. Conversely, using the other forests to calibrate the model leads to a larger underestimate of the Amazon forest fraction.

Finally, we perform a history matching exercise using credible estimates for simulator discrepancy and observational uncertainty terms. We are unable to constrain the parameters individually, but just under half of joint parameter space is ruled out as being incompatible with forest observations. We discuss the possible sources of the discrepancy in the simulated Amazon, including missing processes in the land surface component, and a bias in the climatology of the Amazon.

\end{document}